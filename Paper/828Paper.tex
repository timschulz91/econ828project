\documentclass[12pt]{article}

\usepackage{fullpage}
\usepackage{amsmath}
\usepackage{enumerate}
\usepackage{setspace}
\usepackage{bbm}
\usepackage{graphicx}
\usepackage{hyperref}
\usepackage{framed}
\usepackage{changepage}


\title{Econ 828 Term Project:\\ How Much Is Too Much?\\
		Beliefs About Perceived Inequality In A Coordination Game}
\author{Tim Schulz}
\date{Summer 2016}

\begin{document}
	\maketitle
	\doublespacing
	\section{Introduction}
	Throughout history, there have been numerous examples of how excessive 
	inequality lead people to do something about it at possibly great personal 
	risks while at other times seemingly high inequality in a population was 
	being tolerated. Of course, a lot of these differences can be explained by 
	political systems and forms of government.
	
	I would like to study the question when people perceive inequality as too 
	great in a controlled lab environment. I am interested not only in 
	discovering when ``poor" participant are willing to take action over 
	inequality but also when ``rich" participants think they might be at risk. 
	To achieve this, I am proposing a modified version of the classical stag 
	hunt game spun into a story of a possible revolution.
	
	First, any such revolution's success highly depends on cooperation among 
	the poor. Additionally, inciting an uprising in the real world is very 
	costly if others do not support it. For these reasons, the stag hunt game 
	with a payoff dominant strategy (start a revolution) and a risk dominant 
	strategy (do nothing) provides an excellent representation of the 
	underlying problem. Furthermore, a revolution, successful or not, 
	oftentimes results in a reduction of overall welfare, so in most cases, 
	possible gains to revolutionaries outweigh losses to the rich.
	
	In each stage of the game, players are randomly assigned to groups of three 
	consisting of two poor and one rich participant. Tables that show payoffs 
	dependent on outcomes for rich and poor players are shown to all 
	participants. Based on these tables, each subject knows the earning 
	potentials of other subjects and can see payoff differences between poor 
	and rich players. These differences simulate inequality in the three-person 
	society and vary with treatments.
	
	Poor players decide whether they want to revolt or not. This measures their 
	perception of inequality and whether they think it likely that others will 
	support them. Rich players can decide to purchase insurance against 
	unsuccessful uprisings. This measures what the rich person perceives as 
	acceptable or non-acceptable levels of inequality.
	
	This way, there is also some second-order reasoning involved. Risky 
	revolting might not be worth it if I believe insurance has been purchased, 
	but without insurance I might believe success is more likely.
	I expect to observe four categories of outcomes: (1) inequality is low, 
	poor players choose not to revolt and the rich player does not purchase 
	insurance, (2) inequality is slightly higher so that poor players face only 
	moderate risk since the rich player is (correctly) assumed to not have 
	insurance, (3) inequality is high enough for the rich player to buy 
	insurance which makes a revolution, in expectation, too risky for poor 
	players, and (4) inequality is extreme and the rich player purchases 
	insurance in case a revolution fails but, most of the time, poor players 
	can successfully coordinate on a revolution.
	
	Finally, beliefs about perceived inequality and the likelihood that other 
	players decide to revolt/buy insurance is measured. These beliefs are 
	analysed for a total of six treatments (low/high 
	risk)$\times$(low/medium/high inequality) and separated into whether the 
	belief is a poor or a rich person's perspective (i.e. was the subject 
	assigned the role of a poor or rich player in that round).
	
	\section{Literature Review}
	The stag hunt, being one of the oldest games in game theory, has been the 
	subject of much discussion in experimental economics. Battalio et al. 
	(2001) demonstrate the effect of varying differences between the payoff in 
	the payoff-dominant and the risk-dominant strategies on participants' 
	probability of playing either of the strategies. They find that the 
	likelihood of a player playing the payoff-dominant strategy is decreasing 
	in the size of the optimization premium (the difference between best and 
	inferior response to the opponent's strategy).
	
	
	Rankin et al. (2000) expand on this work and show that subjects' 
	coordination in a \emph{repeated} stag hunt game with varying payoffs 
	always converges to the payoff-dominant equilibrium.
	
	Additionally, adding negative externalities to a cooperation game has been 
	shown to reduce the likelihood of successful cooperation. Engel and 
	Zhurakhovska (2014) find that, if a negative externality to an outsider 
	results from cooperation in a prisoners' dilemma, the two players in that 
	game are less likely to cooperate. Specifically, framed in the context of 
	an oligopoly, they conclude that the degree of cooperation is monotonically 
	decreasing in the harm inflicted on the third party. The suggested rational 
	explanation is one of guilt aversion: Subjects are less likely to increase 
	their profits through cooperation if that additional profit also results in 
	a loss to others, which they may feel guilty about.
	
	So a stag hunt game in itself would not contribute to the literature 
	significantly. However, to the best of my knowledge, a stag hunt with a 
	(possible) negative externality to a third party who can influence the 
	payoffs of the players in the coordination game  in the specific context of 
	inequality has not been investigated in an experiment yet. It remains to be 
	seen if framing the game as one about inequality affects subjects' behavior 
	or if the previously discussed, known results largely hold.
	
	
	\section{Theoretical Framework}
	\subsection{Payoffs}
	The underlying model of the coordination game in this paper is that of a 
	stag hunt with the added dimension of a third party whose decisions 
	potentially affect the payoffs of the other two players who try to 
	coordinate. In the context of inequality, the poor are the parties that 
	play the traditional stag hunt game while the rich person has influence 
	over the exact form of the game by deciding whether to purchase insurance. 
	The resulting payoffs for poor players are shown in Table 
	\ref{table:gpayoff}.
	
	\begin{table}[!htbp]
		\caption{General Payoffs of Poor}
		\label{table:gpayoff}
		\begin{center}
		\begin{tabular}{|l|c|c|c|l|c|c|}
			\multicolumn{3}{c}{Under \texttt{no insurance} $(n)$} &
			\multicolumn{1}{c}{} &
			\multicolumn{3}{c}{Under \texttt{insurance} $(i)$}\\
			\cline{1-3}\cline{5-7}
			& revolt & do nothing & & & revolt & do nothing\\
			\cline{1-3}\cline{5-7}
			revolt & $h_n, h_n$ & $l_n, m_n$ && revolt & $h_i, h_i$ & $l_i, 
			m_i$\\
			\cline{1-3}\cline{5-7}
			do nothing & $m_n, l_n$ & $m_n, m_n$ && do nothing & $m_i, l_i$ & 
			$m_i, m_i$\\
			\cline{1-3}\cline{5-7}
		\end{tabular}
		\end{center}
		\footnotesize
		Payoffs of poor players depending on the other poor player's decision 
		and depending on whether the rich person opted for insurance. 
	\end{table}
	
	In order to correctly represent the classical stag hunt game, it has to be 
	the case that the high payoff for successfully coordinating on the payoff 
	dominant strategy is greater than the medium payoff from 
	playing the risk dominant strategy, which in turn is greater than the 
	payoff from unsuccessfully playing the risky strategy. That is, $h_j > m_j 
	> l_j$ in both the \texttt{no insurance} and the \texttt{insurance} case 
	(i.e. $\forall j\in\{n, i\}$). Additionally, successful coordination on the 
	risky action results in the same payoff regardless of the actions of the 
	rich player ($h_n=h_i=h$) and not revolting is risk free not only within a 
	game but also across the two different games resulting from the rich's 
	decisions ($m_n=m_i=m$). Lastly, revolting unsuccessfully is worse if the 
	rich player takes out insurance ($l_i<l_n$) and can be seen as the rich 
	person ``fighting back". These assumptions yield a simplified payoff table 
	for poor players shown in Table \ref{spayoff}.
	
	\begin{table}[!htbp]
		\caption{Simplified Payoffs of Poor}
		\label{spayoff}
		\centering
		\begin{tabular}{|l|c|c|}
			\hline
			& revolt & do nothing\\
			\hline
			revolt & $h, h$ & $l_j, m$\\
			\hline
			do nothing & $m, l_j$ & $m, m$\\
			\hline
		\end{tabular}\\
		\footnotesize Where $j\in\{i, n\}$ and $l_n>l_i$.
	\end{table}
	
	The decision made by the rich player is that of purchasing insurance 
	against a possible revolution. His payoffs depend on whether a revolution 
	was attempted and, if so, whether it was successful. Since a revolutions is 
	only successful if both poor players decide to play the risky action of 
	``revolt" , this case corresponds to the outcome with two revolutionaries. 
	An outcome with only one revolutionary represents an unsuccessful 
	revolution and zero revolutionaries means nobody tried to revolt.
	
	If a revolutions is successful, having insurance has no effect on the 
	rich's payoffs. That is, in terms of Table \ref{rpayoffs}, 
	$y_{2i}=y_{2n}=y_2$. Instead, insurance is only effective in the case of no 
	or failed revolutions (here denoted by the subscript $f$): 
	$y_{0i}=y_{1i}=y_{fi}$. This is better than the outcome $y_{1n}$, which 
	represents an attempted but unsuccessful revolution and can be thought of 
	as being damaging to the rich nonetheless. The best possible outcome for 
	the rich player is not buying insurance and none of the poor attempting a 
	revolution resulting in $y_{0n}$.\footnote{In summary: $y_{0n} > 
	y_{0i}=y_{1i}=y_{fi} > y_{1n} > y_{2i}=y_{2n}=y_2$} Purchasing insurance 
	can therefore be thought of as costing $y_{0n}-y_{fi}$ and paying out 
	$y_{fi}-y_{1n}$ in the case of a failed revolution attempt.
	
	\begin{table}[!htbp]
		\caption{Payoffs of Rich}
		\label{rpayoffs}
		\centering
		\begin{tabular}{|c||c|c|c|c||c|c|}
			\multicolumn{3}{c}{General Form} &
			\multicolumn{1}{c}{} &
			\multicolumn{3}{c}{Simplified Form}\\
			\cline{1-3}\cline{5-7}
			number of & & no & & number of & & no\\
			revolutionaries & insurance & insurance && revolutionaries & 
			insurance & insurance\\
			\cline{1-3}\cline{5-7}
			0 & $y_{0i}$ & $y_{0n}$ && 0 & $y_{fi}$ & $y_{0n}$\\
			\cline{1-3}\cline{5-7}
			1 & $y_{1i}$ & $y_{1n}$ && 1 & $y_{fi}$ & $y_{1n}$\\
			\cline{1-3}\cline{5-7}
			2 & $y_{2i}$ & $y_{2n}$ && 2 & $y_2$ & $y_2$\\
			\cline{1-3}\cline{5-7}
		\end{tabular}
	\end{table}
	
	\subsection{Decision Making Process}
	Given these payoffs, a rational poor player will choose to revolt 
	conditional on beliefs about what the other poor player will do and whether 
	the rich person will buy insurance. Let a poor player believe that the 
	other poor player chooses to revolt with probability $\gamma$ and that the 
	rich player purchases insurance with probability $\delta$. Then a payoff 
	maximizing poor person will revolt if
	$$\gamma h + (1-\gamma)\left[ (1-\delta)l_n + \delta l_i \right] \geq m $$
	
	A rich person will decide to purchase insurance based on his beliefs about 
	how many poor players will choose to revolt. In the most general case, let 
	a rich player believe that there are zero revolutionaries with probability 
	$\alpha$, one revolutionary with probability $\beta$ and two 
	revolutionaries with probability $1-\alpha-\beta$. Then, a rich player will 
	purchase insurance if
	\begin{align*}
		(\alpha+\beta)y_{fi} + (1-\alpha-\beta)y_2 &\geq \alpha y_{0n} + \beta 
		y_{1n} + (1-\alpha-\beta)y_2\\
		(\alpha+\beta)y_{fi} &\geq \alpha y_{0n} + \beta y_{1n}\\
	\end{align*}
	Assuming independent and identical poor players, $\alpha=(1-\gamma)^2$, 
	$\beta=2\gamma(1-\gamma)$ and $(1-\alpha-\beta)=\gamma^2$. Using this 
	simplification, rich players buy insurance if
	$$ \left[(1-\gamma)^2 + 2\gamma(1-\gamma)\right]y_{fi} \geq 
	(1-\gamma)^2y_{0n} + 2\gamma(1-\gamma)y_{1n} $$
	which simplifies to
	\begin{equation}\tag{I}
		\label{I}
		\gamma \geq \frac{y_{0n} - y_{fi}}{y_{fi} + y_{0n} -2y_{1n}} \equiv 
		\gamma_1.
	\end{equation}
	
	For the decision making process of poor players, this means that if their 
	belief of the other poor player's probability of revolting satisfies 
	condition (\ref{I}), they believe that the rich player will purchase 
	insurance for sure (i.e. $\delta=1$). Considering these second order 
	beliefs, given that $\gamma>\gamma_1$, the poor player chooses to revolt if
	\begin{align*}
		\gamma h + (1-\gamma)l_i &\geq m\\
		\gamma &\geq \frac{m-l_i}{h-l_i} \equiv \gamma_2. \tag{RI}\label{RI}
	\end{align*}
	
	If the poor player believes the other poor player to be sufficiently 
	unlikely to revolt so that the rich player does not purchase insurance 
	(condition (\ref{I}) is not satisfied and $\gamma<\gamma_1$), he will only 
	revolt if
	\begin{align*}
		\gamma h + (1-\gamma)l_n &\geq m\\
		\gamma &\geq \frac{m-l_n}{h-l_n} \equiv \gamma_3. \tag{RN}\label{RN}
	\end{align*}
	
	Combining conditions (\ref{I}), (\ref{RI}) and (\ref{RN}), poor players will
	\begin{enumerate}
		\item revolt under insurance if $\gamma \geq \max (\gamma_1, \gamma_2)$
		\item not revolt due to the presence of insurance if $\gamma_1 \leq 
		\gamma < \gamma_2$
		\item revolt under no insurance if $\gamma_3 \leq \gamma < \gamma_1$
		\item not revolt under no insurance if $\gamma < \min (\gamma_1, 
		\gamma_3)$
	\end{enumerate}
	
	\subsection{Possible Ranges of Outcomes}
	Since none of the conditions (\ref{I}), (\ref{RI}) and (\ref{RN}) 
	contradict each other directly, the ranges of $\gamma$ described above can 
	be arranged in various ways by choosing the payoffs of the rich and poor 
	players appropriately. Depending on payoff, some of the cases above can be 
	ruled out from occurring. Possible (rational) outcomes are
	\begin{enumerate}[I.]
		\item 	$0 \leq \gamma_3 < \gamma_1 < \gamma_2 \leq 1$\\
				This is the only arrangement in which all four cases could 
				potentially be observed. With low beliefs about the probability 
				of revolution ($\gamma<\gamma_3$), the rich person does not 
				purchase insurance and poor players do not risk revolting. For 
				a slightly higher perceived probability 
				($\gamma_3\leq\gamma<\gamma_1$), it is still not worth 
				purchasing insurance for the rich player, but poor players 
				still try to revolt. For $\gamma_1\leq\gamma<\gamma_2$, the 
				rich player buys insurance which in turn deters poor players 
				from attempting a revolution as the consequences of failure are 
				now graver than they were without insurance. If a revolution is 
				believed to be likely ($\gamma\geq\gamma_2$), the rich player 
				still purchases insurance on the off-chance that the revolution 
				fails (if the revolution is successful, the outcome for the 
				rich person is the same regardless of insurance) and poor 
				players decide to revolt.
		\item	$0 \leq \gamma_3 < \gamma_2 < \gamma_1 \leq 1$\\
				The rich player does not purchase insurance for values of 
				$\gamma$ below $\gamma_1$. If poor players also believe 
				$\gamma<\gamma_3$, they will not revolt even in the absence of 
				insurance. For $\gamma_3\leq\gamma_1$, poor revolt without 
				facing the rich's insurance. If the beliefs exceed $\gamma_1$, 
				the rich player buys insurance and the poor revolt.
		\item	$0 \leq \gamma_2 < \gamma_3 < \gamma_1 \leq 1$\\
				This case allows for identical outcomes as the one above since 
				$\gamma_2$ is meaningless if $\gamma_2<\gamma_1$.
		\item	$0 \leq \gamma_1 < \gamma_3 < \gamma_2 \leq 1$\\
				There will be no insurance as long as players believe the 
				probability of one poor player to revolt to be less than 
				$\gamma_1$. For values above $\gamma_1$, the rich player buys 
				insurance which deters poor players from revolting as long as 
				beliefs about $\gamma$ are such that $\gamma<\gamma_2$. If the 
				latter condition is not satisfied, the outcome will be a 
				revolution under insurance.
		\item	$0 \leq \gamma_1 < \gamma_2 < \gamma_3 \leq 1$\\
				This is, again, equivalent to the case above as $\gamma_3$ 
				carries no meaning if $\gamma_3 > \gamma_1$.
		\item	$0 \leq \gamma_2 < \gamma_1 < \gamma_3 \leq 1$\\
				This last case only allows for two possible rational outcomes: 
				No 
				insurance and no revolution if $\gamma<\gamma_1$ or revolution 
				in the presence of insurance if the opposite is believed to be 
				true.
	\end{enumerate}
	
	\section{Experimental Design}
	The experiment follows a $2\times2\times3$ design. First, subjects are 
	randomly assigned either the role of a poor or a rich player. Second, 
	payoffs are designed to have either high or low variation in payoffs for 
	each of the players. Lastly, payoffs are arranged such that the inequality 
	between possible outcomes for poor and rich players is either high, medium 
	or low.
	
	In order to allow for all possible rational actions for poor players as 
	discussed in section 3.2, payoffs are chosen such that they meet the 
	requirements of case I. in section 3.3. These payoffs can be seen in Table 
	\ref{allpayoffs}.
	
	\begin{table}[!htbp]
		\caption{Payoff Treatments}
		\label{allpayoffs}
		\centering
		\begin{tabular}{|c|c||c|c|c|c|c|c|c|}
			\multicolumn{2}{c}{Treatment} &\multicolumn{7}{c}{Payoffs}\\
			\hline
			Risk & Inequality & $l_i$ & $l_n$ & $m$ & $h$ & $y_{1n}$ & $y_{fi}$ 
			& $y_{on}$\\
			\hline
			low & low & 130 & 280 & 440 & 750 & 490 & 1100 & 1820\\
			low & medium & 130 & 280 & 440 & 750 & 980 & 2200 & 3640\\
			low & high & 130 & 280 & 440 & 750 & 1960 & 4400 & 7280\\
			high & low & 75 & 130 & 320 & 750 & 500 & 2250 & 4000\\
			high & medium & 75 & 130 & 320 & 750 & 1000 & 4500 & 8000\\
			high & high & 75 & 130 & 320 & 750 & 2000 & 9000 & 16000\\			
			\hline
		\end{tabular}
	\end{table}
	
	Subjects are not informed about the existence of these six treatments. 
	Instead, they only know that there are rich and poor players assigned 
	randomly and that two poor players are randomly paired with one rich player 
	into one group.
	
	All players are shown the payoffs selected for the current round. 
	The values from Table \ref{allpayoffs} are filled into Table \ref{spayoff} 
	and into the right side of Table \ref{rpayoffs}. All players see both 
	tables. That way, poor players not only know the risk that they (and the 
	other poor player) face but also how unequal their payoffs are relative to 
	the rich player. Additionally, poor players can see how risky the rich 
	player's possible payoffs are. Based on this information, poor players form 
	beliefs about the the likelihood of the other poor player revolting and the 
	rich player purchasing insurance (which determines which payoff table is 
	relevant for poor players). 
	
	In a similar manner, rich players see their own payoffs and the range 
	between highest and lowest possible outcome on top of how much more they 
	can expect to earn relative to the two poor players in the group and how 
	risky the latter's payoffs are.
	
	Next, poor (rich) players choose whether to revolt (buy insurance). 
	Additionally, rich players are asked to guess whether a revolution will be 
	successful. They are told that guessing correctly is rewarded with 100 
	points.
	Afterwards, all players are informed about how many poor players decided to 
	revolt\footnote{From this, poor players can infer if the other poor player 
	revolted and learn about the likelihood of a revolution based on observed 
	payoffs.} and if insurance was purchased, and all players are told their 
	payoffs from that round.
	
	At the end of a round, each subject is randomly assigned to the role of 
	either a poor or rich player and is furthermore randomly assigned to a new 
	group. Randomizing the order in which subjects play rich or poor players, 
	are faced with high or low risk payoffs, and observe high, medium or low 
	inequality, allows me to ignore any possible order effects.
	
	At the conclusion of the experiment, subjects are paid for one randomly 
	selected period. Monetary earnings are calculated as $\frac{1}{100}$ 
	times the number of points earned in the selected round plus a show-up fee 
	of \$7.
	
	\section{Analysis}
	The focus of this experiment is the subject's belief over the likelihood of 
	successful coordination in response to observed inequality. In the case of 
	poor players, these beliefs are directly observable: A subject should 
	revolt if and only if the other poor player is believed to choose revolt. 
	Therefore, observing a subject's frequency of choosing to revolt, directly 
	represents their belief about the choice of the other poor player. In the 
	case of rich players, this analysis is somewhat less straight-forward. 
	Instead of directly observing the frequency of revolutions, the rich person 
	only indicates whether he believes in a successful revolution. This, 
	however, does not translate directly into $\gamma$ as the probability of 
	both poor players choosing to revolt is $\gamma^2$ as stated previously.
	Assuming that the rich player believes revolutions to be successful with 
	probability $\tau$, the variable of interest is $\sqrt{\tau} = 
	\gamma$. 
	In addition, it is possible to 
	identify a range that the rational rich player must have believed the 
	probability of a revolution to fall into. Using condition (\ref{I}), it is 
	possible to conclude that the rich player must think that $\gamma>\gamma_1$ 
	if he purchased insurance while the opposite must be true if he did not 
	purchase insurance. This range can be used to verify the validity of using 
	$\sqrt{\tau}$ as the dependent variable in the analysis.
	
	\subsection{Regression Analysis}
	After observing $\gamma$ and $\sqrt{\tau}=\gamma$ for each subject (the 
	former for in cases when subjects were in the role of a poor player and the 
	latter in the role of the rich player) and treatment,\footnote{Assigning 
	the value 1 whenever a poor (rich) person chooses to revolt (buy insurance) 
	and 0 otherwise, the probabilities are the means within a treatment for 
	each 
	subject. This results in six observations per subject.} the goal is to 
	estimate how inequality affects a subject's belief over the probability of 
	a revolution while controlling for risk in the payoffs. The second effect 
	of interest is how perspective changes beliefs or the perception of 
	inequality. This can be quantified as the effect of being assigned either 
	role.

	An appropriate dummy variable model is therefore
	
	\begin{align*}
		\gamma_{it} = b_0 &+ b_1\mathbbm{1}\{inequality_t=medium\}\\
		&+ b_2\mathbbm{1}\{inequality_t=high\}\\
		&+ b_3\mathbbm{1}\{role_t=rich\}\\
		&+ b_4\mathbbm{1}\{risk_t=high\} + \kappa_i + \epsilon_{it}
	\end{align*}
	
	where $b_0$ is the average belief of a poor person playing the game in a 
	low inequality and low risk setup. $b_1$ and $b_2$ identify the effects of 
	of medium and high inequality, respectively. $b_3$ measures how a subject's 
	perspective changes as he switches from the role of a poor player to that 
	of a rich player. The dummy variable whose coefficient is $b_4$ controls 
	for possible risk preferences and $\kappa_i$ captures unobserved 
	heterogeneity among subjects. For example, some subjects might be biased 
	towards thinking that inequality is inherently bad and are therefore more 
	likely to choose to revolt.
	
	Preferably, this model would be estimated using OLS in order to identify 
	linear effects. However, should predicted values of $\gamma_{it}$ fall 
	outside of $[0,1]$, a more appropriate Tobit model would have to 
	be used.
	
	\subsection{Initial Hypotheses}
	A strong positive effect of higher inequality should be observable. Since 
	poor people need to coordinate on one actions, anything from history to 
	social norms suggests that subjects believe that the other subject is more 
	likely to revolt if inequality is greater. Therefore, a reasonable 
	assumption would be that $0<b_1<b_2$.
	
	Next, I would expect rich players to perceive inequality as less pressing 
	than if they observed the same payoff possibilities as a poor person, 
	meaning $b_3<0$. Though theory does not directly predict it, I would also 
	expect $b_3$ to be smaller in magnitude than $b_2$.
	
	Lastly, while the level of risk only serves as a control variable here, a 
	brief discussion of its effect is in order. Risk, as measured in this 
	experiment, only concerns the relative difference between the best and 
	worst possible payoff for a poor player. It is therefore unclear if risk is 
	perceived as the size of a possible loss ($m-l_i$ if insurance is purchased 
	or $m-l_n$ otherwise) or the size of possible gains ($h-m$). The direction 
	of this effect is therefore not clear. The reason risk is a treatment 
	variable in this experiment, is not to measure risk preferences, but rather 
	to control for any possible reaction to risk left after (technically) 
	inducing risk-neutrality by choosing a random round for payment and by 
	doing so to get the least polluted measure of the effect of inequality.
	
	\section{Simulation Exercise}
	A crude implementation in a Monte Carlo simulation\footnote{The script 
	underlying this simulation can be found at 
	\url{https://github.com/timschulz91/econ828project/blob/master/Simulation.ipynb}}
	 of the basics of 
	the 
	model presented here verifies the relevance of the proposed research 
	question. The simulation varies from the experiment in a few important 
	aspects. First, ``subjects" (or in this case class instances) do not switch 
	between the roles of poor and rich players. Instead, any initial 
	assignments persist throughout the simulation. Second, the six remaining 
	treatments are not directly implemented. Instead, there is a (near) 
	continuum of treatments. At the beginning of the simulation, 10,000 payoff 
	tables are randomly generated. Of these 10,000, 1,707 satisfied condition 
	the condition that $0 \leq \gamma_3 < \gamma_1 < \gamma_2 \leq 1$, which is 
	necessary to rationally allow for all possible combinations of revolt 
	$\times$ insurance.
	
	Throughout 500 rounds, 30 subjects learned and formed beliefs about the 
	probability of a revolution in response to a variable $d$, which measures, 
	in very simple terms, the degree of inequality and is defined as one minus 
	the ratio of a poor person's payoff relative to that of a rich person if 
	the status quo remained (i.e. nobody revolts and the rich person does not 
	purchase insurance). That is $d=1-\frac{m}{y_{0n}}$. In each round, 
	subjects were randomly assigned into groups of two poor and one rich 
	player. Then, all players were informed about the degree of inequality $d$ 
	in response to which poor players predicted $\gamma$ as a linear function 
	of $d$ and decided to revolt with that probability $\gamma$. Rich players 
	were simulated to purchase insurance if they predicted (based on the Random 
	Forest machine learning algorithm) at least one poor player to revolt. At 
	the end of a round, all simulated players were informed about how many poor 
	players revolted and whether insurance was purchased.
	
	\begin{figure}[!htbp]
		\caption{Simulation Graph}
		\label{simgraph}
		\centering
		\includegraphics[width=.5\textwidth]{../graph.png}
	\end{figure}
	
	The results of the simulation can be seen in Figure \ref{simgraph}. The 
	graph depicts the observed frequency of poor players choosing to revolt in 
	response to the simplified measure of inequality $d$. As predicted by 
	theory, there is a generally positive relationship between inequality and a 
	poor player's propensity to choose to revolt. The notable exception is the 
	decrease for observations with $d\approx0.6$. This, again, corresponds to 
	theory in the sense that theory predicts a range in which inequality is 
	high enough that rich players purchase insurance, which in turn deters poor 
	players from revolting. Only once inequality keeps increasing, poor players 
	can again coordinate on the risky action of revolting.
	
	
	\section{Conclusion}
	In conclusion, this experiment should be useful in answering several 
	questions regarding inequality. The first question is the question already 
	raised in the title: How much is too much? At what degree of inequality are 
	poor people in a society willing to rise up against the status quo despite 
	personal risk, known negative consequence for the rich, and known decline 
	of overall welfare? The obvious null hypothesis here is that higher 
	inequality increases the probability (and beliefs) of a revolution being 
	successful.
	
	Second, how is inequality perceived differently depending on whether a 
	person is rich or poor? A simple hypothesis would be that rich people, on 
	average, believe observed inequality to be less severe than poor people. 
	This holds even though rich subjects are forced to anticipate what the 
	poor will do, which makes them view inequality through the lens of a poor 
	person. Another question that is interesting in this context is whether 
	this distortion of perspective is increasing or decreasing in the degree of 
	inequality.
	
	Several possible extension to this experiment come to mind. The next step 
	should be controlling for the type of payments to player. The way it is 
	proposed in this experiment, players receive endowments of a certain number 
	of points. This may be viewed as too arbitrary. Poor subjects may not 
	consider the negative externality of their actions on rich subjects as 
	significant since points are just handed out randomly. This view may change 
	if points are earned instead. Poor players may be less likely to revolt if 
	they think that the rich player worked for his points and, thus, may 
	actually 
	deserve the points. An experimental setup like this would also better 
	represent reality\footnote{Differences in income or wealth are 
	earned/deserved in most cases (even though this could ultimately depend on 
	political ideology).} but it would add an additional treatment to this 
	experiment which is already designed relatively complex as 
	$2\times3\times3$.
	
	Next, it may be necessary to further emphasize the risk of choosing to 
	revolt. In the real world, revolutions do not normally lead to one of two 
	outcomes, but the outcomes can vary significantly. Specifically, a 
	successful revolution can still lead to less desirable conditions for the 
	poor than those that were in place before the revolution in the sense that 
	too much of the overall welfare has been destroyed. To capture this, one 
	modification to the experiment would be to make payoffs for the action 
	revolt completely random (beyond merely being dependent on other players' 
	actions) in some kind of lottery.
	
	Lastly, it may be worth controlling for the context. This experiment 
	specifically aims to reveal information about people's perception of 
	inequality. However, subjects in the experiment may make decisions based on 
	their attitude towards inequality as well as profit maximization. 
	Administering the same experiment but taken out of the context of 
	inequality 
	and the roles of rich and poor could provide a baseline case that describes 
	the behaviour of people who are solely driven by the principle of 
	maximization.
	
	\begin{thebibliography}{99}
		\bibitem{Battalio2001}
		Battalio, R., Samuelson, L., and Van Huyck, J. (1957), ``Optimization 
		Incentives and Coordination Failure in Laboratory Stag Hunt Games",
		\textit{Econometrica}, Vol. 69-3, pp. 749--764.
		
		\bibitem{Engel2012}
		Engel, C., and Zhurakhovska, L. (2014), ``Conditional Cooperation
		With Negative Externalities -- An Experiment", \emph{Journal of 
		Economic Behavior \& Organization}, 108, pp. 252--260.
		
		\bibitem{Rakin2000}
		Rakin, F., Van Huyck, J., and Battalio, R. (2000), ``Strategic 
		Similarity and Emergent Conventions: Evidence from Similar Stag Hunt 
		Games", \emph{Games and Economic Behavior}, 32, pp. 315--337.
	\end{thebibliography}
	
	\newpage
	\section*{Appendix}
	\subsection*{Instructions}
	Welcome to the experiment and thank you for participating!
	
	If you follow these instructions, you can earn a considerable amount of 
	money paid out in cash at the end of the experiment. Your earnings will 
	depend on decisions you make during the course of this experiment and also 
	on decisions that other participants make.
	
	During the experiment, you must remain seated and are not allowed to 
	communicate with other people in the room in any way. Please turn of any 
	electronic devices and direct your attention solely at these instructions 
	and the computer screen in front of you. If you have questions at any time, 
	please raise your hand and an experimenter will come and answer your 
	questions privately.
	
	\subsubsection*{The Experiment}
	This experiment consists of 30 rounds plus 5 practice rounds in the 
	beginning. Once the reading of these instructions concludes, the first 
	practice round begins.
	
	In each round, you will be randomly assigned into a group with two other 
	participants. The computer will randomly decide whether you are rich or 
	poor. Each group consists of two poor and one rich person. For example, if 
	you are determined to be poor, one other participant in your group will be 
	poor as well while the third participant will be rich. If you are 
	determined to be rich, both of the other two participants in your group are 
	poor.
	
	In the beginning of a round, you will receive a number of points.
	
	If you are poor, you have the choice to revolt against the rich person or 
	to not revolt and receive your points. A revolution 
	can only be successful if both poor people decide to revolt. This means, if 
	you decide to revolt, but the other poor participant in your group does 
	not, your revolution will fail. In this case, you will lose points. 
	Additionally, the number of 
	points you lose depends on whether the rich participant purchases 
	insurance against an attempted revolution (more on that later). If you 
	revolt but fail, you lose more points if the rich participant has 
	insurance. You do not lose as many points if your revolution fails when the 
	rich participant does not have insurance.
	If the revolution is successful, both poor participants will receive 
	additional points (on top of the points you were given at the beginning of 
	the round). 
	
	If you are rich, you have the choice to purchase insurance or not to 
	purchase insurance. If you do not purchase insurance and neither of the 
	poor participants in the group decides to revolt, you keep all the points 
	you were given at the beginning of the round. If you do not purchase 
	insurance, and one of the two poor participants chooses to revolt, you lose 
	some of your points. If you decide to purchase insurance, you lose some 
	points (the price of the insurance) but you do not lose as many points as 
	you would without insurance if one of the poor participants revolts. With 
	insurance, you will keep the same amount of points (after paying some 
	points for the insurance) in the case of none or one of the poor 
	participants choosing to revolt. That is, you are insured against failed 
	revolutions. If both poor participants decide to revolt, you will lose all 
	your points and the outcome is the same whether you bought insurance or not.
	
	All participants are shown the number of points they and all other 
	participants would keep for all possible outcomes. Poor participants see 
	how many points they would keep depending on whether the other poor 
	participant decides to revolt and whether the rich participant has 
	insurance. Poor participants also see how many points the rich participant 
	keeps in each outcome. Likewise, if you are rich, you will see all possible 
	payoffs for yourself and for the two poor participants in your group.
	
	Here is an example:
	
	\noindent\rule[0.5ex]{\linewidth}{1pt}
		You are \textbf{poor} this round.\\		
		You have 200 points. The rich participant has 500 points.\\		
		This table shows how many points you will keep at the end of the round 
		depending on what the other poor participant and the rich participant 
		do.
		
	\begin{adjustwidth*}{-2cm}{-2cm}
		\centering
		\begin{tabular}{l|c|c|c|cl|c|c|}
			\multicolumn{8}{c}{The rich participant has...}\\
			\hline
			\multicolumn{3}{c}{... no insurance} && \multicolumn{1}{c}{}&
			\multicolumn{3}{c}{... insurance}\\
			You... &\multicolumn{2}{c}{The other poor participant...} 
			&&  & You...
			&\multicolumn{2}{c}{The other poor participant...}\\
			\cline{1-3}\cline{6-8}
			& revolts & does not revolt &&&& revolts & does not 
			revolt\\
			\cline{2-3}\cline{7-8}
			revolt & 300, 300 & 100, 200 &&& revolt & 300, 300 & 0, 200\\
			\cline{1-3}\cline{6-8}
			do not revolt & 200, 100 & 200, 200 &&& do not revolt &200, 0 & 
			200, 200\\
			\cline{2-3}\cline{7-8}
		\end{tabular}
	\end{adjustwidth*}
	
	\bigskip
	For example, if the rich participant purchases insurance and the other poor 
	participant does not revolt but you do revolt, you will keep 0 of your 200 
	points (you lose 200 points). Alternatively, if the rich participant does 
	not have insurance and both you and the other poor participant decide to 
	revolt, you will receive an additional 100 points for a total of 300. In 
	general, depending on the outcome, the first number in each cell shows the 
	number of point you will have at the end of the round should that outcome 
	occur. The second number shows the number of points that the other poor 
	participant would end the round with.
	
	This table shows how many points a rich participant (who is given 500 
	points) will have at the end 
	of the round depending on his or her decision to buy insurance and your and 
	the other poor participant's decision to revolt or not to revolt.
	
	\begin{center}

	\begin{tabular}{|c||c|c|}
		\hline
		number of & & no\\
		revolutionaries & insurance & insurance\\
		\hline
		0 & 300 & 500\\
		\hline
		1 & 300 & 200\\
		\hline
		2 & 0 & 0\\
		\hline
	\end{tabular}
	\end{center}
	
	For example, if nobody decides to revolt and the rich participant does not 
	purchase insurance, he or she will keep all 500 points. If you but not the 
	other poor participant (or the other way around) decide to revolt, so that 
	only one participant in total is revolting, the rich participant will keep 
	300 points with insurance or 200 points without insurance. If both you and 
	the other participant revolt, it does not matter if the rich participant 
	has insurance and he or she will lose all of his or her points.
	
	\noindent\rule[0.5ex]{\linewidth}{1pt}
	
	Lastly, if you are the rich participant, you will be asked to indicate 
	whether 
	you believe that a revolution will be successful this round (meaning both 
	poor participants decide to revolt). You will be asked this at the same 
	time when you are asked whether you would like to buy insurance.
	
	At the end of each round, all players are informed about the number of 
	revolutionaries and whether the rich participant purchased insurance. This 
	way, if you are a poor player that round and you revolted but the total 
	number of revolutionaries is only 1, you know that the other poor player 
	did not revolt. Next, you are informed about how many point you keep from 
	that round. This concludes a round and participants are randomly reassigned 
	for the next round.
	
	\subsubsection*{Your Payment}
	Every participant will receive a \$7 show-up payment.
	
	Additionally, you will be paid for one randomly chosen round. You will be 
	paid in cash and receive \$1 per 100 points you kept at the end of the 
	randomly selected round. If you were rich in this round, you will also 
	receive a \$1 bonus if you correctly guessed how many of the poor 
	participants revolted.
	
	All your decisions and the resulting payments are confidential and we do 
	not record your name.
	
	\bigskip
	This concludes the instructions. If you have any questions, please raise 
	your hand. Thank you for participating in this experiment!
	
\end{document}
