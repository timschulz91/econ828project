\documentclass[12pt]{article}

\usepackage{fullpage}
\usepackage{amsmath}
\usepackage{enumerate}
\usepackage{setspace}


\title{Econ 828 Term Project:\\ How much is too much?\\
		Beliefs about perceived inequality in a coordination game}
\author{Tim Schuz}
\date{Summer 2016}

\begin{document}
	\maketitle
	\doublespacing
	\section{Introduction}
	\section{Litereature Review}
	\section{Theoretical Framework}
	\subsection{Payoffs}
	The underlying model of the coordination game in this paper is that of a stag hunt with the added dimension of a third party whose decisions potentially affect the payoffs of the other two players who try to coordinate. In the context of inequality, the poor are the parties that play the traditional stag hunt game while the rich person has influence over the exact form of the game by deciding whether to purchase insurance. The resulting payoffs for poor players are shown in Table \ref{gpayoff}.
	
	\begin{table}[!htbp]
		\caption{General Payoffs of Poor}
		\label{gpayoff}
		\begin{center}
		\begin{tabular}{|l|c|c|c|l|c|c|}
			\multicolumn{3}{c}{Under \texttt{no insurance} $(n)$} &
			\multicolumn{1}{c}{} &
			\multicolumn{3}{c}{Under \texttt{insurance} $(i)$}\\
			\cline{1-3}\cline{5-7}
			& revolt & do nothing & & & revolt & do nothing\\
			\cline{1-3}\cline{5-7}
			revolt & $h_n, h_n$ & $l_n, m_n$ && revolt & $h_i, h_i$ & $l_i, m_i$\\
			\cline{1-3}\cline{5-7}
			do nothing & $m_n, l_n$ & $m_n, m_n$ && do nothing & $m_i, l_i$ & $m_i, m_i$\\
			\cline{1-3}\cline{5-7}
		\end{tabular}
		\end{center}
		\footnotesize
		Payoffs of poor players depending on the other poor player's decision and depending on whether the rich person opted for insurance. 
	\end{table}
	
	In order to correctly represent the classical stag hunt game, it has to be the case that the high payoff for successfully coordinating on the payoff dominant strategy is greater than the medium payoff for playing from playing the risk dominant strategy, which in turn is greater than the payoff from unsuccessfully playing the risky strategy. That is, $h_j > m_j > l_j$ in both the \texttt{no insurance} and the \texttt{insurance} case (i.e. $\forall j\in\{n, i\}$). Additionally, successful coordination on the risky action results in the same payoff regardless of the actions of the rich player ($h_n=h_i=h$) and not revolting is risk free not only within a game but also across the two different games resulting from the rich's decisions ($m_n=m_i=m$). Lastly, revolting unsuccessfully is worse if the rich player takes out insurance ($l_i<l_n$) and can be seen as the rich person ``fighting back". These assumptions yield a simplified payoff table for poor players shown in Table \ref{spayoff}.
	
	\begin{table}[!htbp]
		\caption{Simplified Payoffs of Poor}
		\label{spayoff}
		\centering
		\begin{tabular}{|l|c|c|}
			\hline
			& revolt & do nothing\\
			\hline
			revolt & $h, h$ & $l_j, m$\\
			\hline
			do nothing & $m, l_j$ & $m, m$\\
			\hline
		\end{tabular}\\
		\footnotesize Where $j\in\{i, n\}$ and $l_n>l_i$.
	\end{table}
	
	The decision made by the rich player is that of purchasing insurance against a possible revolution. His payoffs depend on whether a revolution was attempted and, if so, whether it was successful. Since a revolutions is only successful if both poor players decide to play the risky action of ``revolt" , this case corresponds to the outcome with two revolutionaries. An outcome with only one revolutionary represents an unsuccessful revolution and zero revolutionaries mean nobody tried to revolt.
	
	If a revolutions is successful, having insurance has no effect on the rich's payoffs. That is, in terms of Table \ref{rpayoffs}, $y_{2i}=y{2n}=y_2$. Instead, insurance is only effective in the case of no or failed revolutions (here denoted by the subscript $f$): $y_{0i}=y_{1i}=y{fi}$. This is better than the outcome $y_{1n}$, which represents an attempted but unsuccessful revolutions and can be thought of as being damaging to the rich nonetheless. The best possible outcome for the rich player is not buying insurance and none of the poor attempting a revolution resulting in $y_{0n}$.\footnote{In summary: $y_{0n} > y_{0i}=y_{1i}=y_{fi} > y_{1n} > y_{2i}=y_{2n}=y_2$} Purchasing insurance can therefore be thought of as costing $y_{0n}-y_{fi}$ and paying out $y_{fi}-y_{1n}$ in the case of a failed revolution attempt.
	
	\begin{table}[!htbp]
		\caption{Payoffs of Rich}
		\label{rpayoffs}
		\centering
		\begin{tabular}{|c||c|c|c|c||c|c|}
			\multicolumn{3}{c}{General Form} &
			\multicolumn{1}{c}{} &
			\multicolumn{3}{c}{Simplified Form}\\
			\cline{1-3}\cline{5-7}
			number of & & no & & number of & & no\\
			revolutionaries & insurance & insurance && revolutionaries & insurance & insurance\\
			\cline{1-3}\cline{5-7}
			0 & $y_{0i}$ & $y_{0n}$ && 0 & $y_{fi}$ & $y_{0n}$\\
			\cline{1-3}\cline{5-7}
			1 & $y_{1i}$ & $y_{1n}$ && 1 & $y_{fi}$ & $y_{1n}$\\
			\cline{1-3}\cline{5-7}
			2 & $y_{2i}$ & $y_{2n}$ && 2 & $y_2$ & $y_2$\\
			\cline{1-3}\cline{5-7}
		\end{tabular}
	\end{table}
	
	\subsection{Decision Making Process}
	Given these payoffs, a rational poor player will choose to revolt 
	conditional on beliefs about what the other poor player will do and whether 
	the rich person will buy insurance. Let a poor player believe that the 
	other poor player chooses to revolt with probability $\gamma$ and that the 
	rich player purchases insurance with probability $\delta$. Then a payoff 
	maximizing poor person will revolt if
	$$\gamma h + (1-\gamma)\left[ (1-\delta)l_n + \delta l_i \right] \geq m $$
	
	A rich person will decide to purchase insurance based on his beliefs about 
	how many poor players will choose to revolt. In the most general case, let 
	a rich player believe that there are zero revolutionaries with probability 
	$\alpha$, one revolutionary with probability $\beta$ and two 
	revolutionaries with probability $1-\alpha-\beta$. Then a rich player will 
	purchase insurance if
	\begin{align*}
		(\alpha+\beta)Y_{fi} + (1-\alpha-\beta)y_2 &\geq \alpha y_{0n} + \beta 
		y_{1n} + (1-\alpha-\beta)y_2\\
		(\alpha+\beta)Y_{fi} &\geq \alpha y_{0n} + \beta y_{1n}\\
	\end{align*}
	Assuming independent and identical poor players, $\alpha=(1-\gamma)^2$, 
	$\beta=2\gamma(1-\gamma)$ and $(1-\alpha-\beta)=\gamma^2$. Using this 
	simplification, rich players buy insurance if
	$$ \left[(1-\gamma)^2 + 2\gamma(1-\gamma)\right]y_{fi} \geq 
	(1-\gamma)^2y_{0n} + 2\gamma(1-\gamma)y_{1n} $$
	which simplifies to
	\begin{equation}\tag{I}
		\label{I}
		\gamma \geq \frac{y_{0n} - y_{fi}}{y_{fi} + y_{0n} -2y_{1n}} \equiv 
		\gamma_1.
	\end{equation}
	
	For the decision making process of poor players, this means that if their 
	belief of the other poor player's probability of revolting satisfies 
	condition (\ref{I}), they believe that the rich player will purchase 
	insurance for sure (i.e. $\delta=1$). Considering these second order 
	beliefs, given that $\gamma>\gamma_1$, the poor player chooses to revolt if
	\begin{align*}
		\gamma h + (1-\gamma)l_i &\geq m\\
		\gamma &\geq \frac{m-l_i}{h-l_i} \equiv \gamma_2. \tag{RI}\label{RI}
	\end{align*}
	
	If the poor player believes the other poor player to be sufficiently 
	unlikely to revolt so that the rich player does not purchase insurance 
	(condition (\ref{I}) is not satisfied and $\gamma<\gamma1$), he will only 
	revolt if
	\begin{align*}
		\gamma h _ (1-\gamma)l_n &\geq m\\
		\gamma &\geq \frac{m-l_n}{h-l_n} \equiv \gamma_3. \tag{RN}\label{RN}
	\end{align*}
	
	Combining conditions (\ref{I}), (\ref{RI}) and (\ref{RN}), poor players will
	\begin{enumerate}
		\item revolt under insurance if $\gamma \geq \max (\gamma_1, \gamma_2)$
		\item not revolt due to the presence of insurance if $\gamma_1 \leq 
		\gamma < \gamma_2$
		\item revolt under no insurance if $\gamma_3 \leq \gamma < \gamma_1$
		\item not revolt under no insurance if $\gamma < \min (\gamma_1, 
		\gamma_3)$
	\end{enumerate}
	
	\subsection{Possible Ranges of Outcomes}
	Since none of the conditions (\ref{I}), (\ref{RI}) and (\ref{RN}) 
	contradict each other directly, the ranges of $\gamma$ described above can 
	be arranged in various ways by choosing the payoffs of the rich and poor 
	players appropriately. Depending on payoff, some of the cases above can be 
	ruled out from occurring. Possible (rational) outcomes are
	\begin{enumerate}[I.]
		\item 	$0 \leq \gamma_3 < \gamma_1 < \gamma_2 \leq 0$\\
				This is the only arrangement in which all four cases could 
				potentially be observed. With low beliefs about the probability 
				of revolution ($\gamma<\gamma_3$), the rich person does not 
				purchase insurance and poor players do not risk revolting. For 
				a slightly higher perceived probability 
				($\gamma_3\leq\gamma<\gamma_1$), it is still not worth 
				purchasing insurance for the rich player, but poor players 
				still try to revolt. For $\gamma_1\leq\gamma<\gamma_2$, the 
				rich player buys insurance which in turn deters poor players 
				from attempting a revolution as the consequences of failure are 
				now graver than they were without insurance. If a revolution is 
				believed to be likely ($\gamma\geq\gamma_2$), the rich player 
				still purchases insurance on the off-chance that the revolution 
				fails (if the revolution is successful, the outcome for the 
				rich person is the same regardless of insurance) and poor 
				players decide to revolt.
		\item	$0 \leq \gamma_3 < \gamma_2 < \gamma_1 \leq 1$\\
				The rich player does not purchase insurance for values of 
				$\gamma$ below $\gamma_1$. If poor players also believe 
				$\gamma<\gamma_3$, they will not revolt even in the absence of 
				insurance. For $\gamma_3\leq\gamma_1$, poor revolt without 
				facing the rich's insurance. If the beliefs exceed $\gamma_1$, 
				the rich player buys insurance and the poor revolt.
		\item	$0 \leq \gamma_2 < \gamma_3 < \gamma_1 \leq 1$\\
				This case allows for identical outcomes as the one above since 
				$\gamma_2$ is meaningless if $\gamma_2<\gamma_1$.
		\item	$0 \leq \gamma_1 < \gamma_3 < \gamma_2 \leq 1$\\
				There will be no insurance as long as players believe the 
				probability of one poor player to revolt to be less than 
				$\gamma_1$. For values above $\gamma_1$, the rich player buys 
				insurance which deters poor players from revolting as long as 
				beliefs about $\gamma$ are such that $\gamma<\gamma_2$. If the 
				latter condition is not satisfied, the outcome will be a 
				revolution under insurance.
		\item	$0 \leq \gamma_1 < \gamma_2 < \gamma_3 \leq 1$\\
				This is, again, equivalent to the case above as $\gamma_3$ 
				carries no meaning if $\gamma_3 > \gamma_1$.
		\item	$0 \leq \gamma_2 < \gamma_1 < \gamma_3 \leq 1$\\
				This last case only allows for two possible outcomes: No 
				insurance and no revolution if $\gamma<\gamma_1$ or revolution 
				in the presence of insurance if the opposite is believed to be 
				true.
	\end{enumerate}
	
\end{document}
